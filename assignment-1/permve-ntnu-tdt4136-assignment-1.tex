%%%%%%%%%%%%%%%%%%%%%%%%%%%%%%%%%%%%%%%%%
% Short Sectioned Assignment
% LaTeX Template
% Version 1.0 (5/5/12)
%
% This template has been downloaded from:
% http://www.LaTeXTemplates.com
%
% Original author:
% Frits Wenneker (http://www.howtotex.com)
%
% License:
% CC BY-NC-SA 3.0 (http://creativecommons.org/licenses/by-nc-sa/3.0/)
%
%%%%%%%%%%%%%%%%%%%%%%%%%%%%%%%%%%%%%%%%%

%----------------------------------------------------------------------------------------
%	PACKAGES AND OTHER DOCUMENT CONFIGURATIONS
%----------------------------------------------------------------------------------------

\documentclass[paper=a4, fontsize=11pt]{scrartcl} % A4 paper and 11pt font size

\usepackage[T1]{fontenc} % Use 8-bit encoding that has 256 glyphs
\usepackage{fourier} % Use the Adobe Utopia font for the document - comment this line to return to the LaTeX default
\usepackage[english]{babel} % English language/hyphenation
\usepackage{amsmath,amsfonts,amsthm} % Math packages

\usepackage{lipsum} % Used for inserting dummy 'Lorem ipsum' text into the template

\usepackage{sectsty} % Allows customizing section commands
\allsectionsfont{\centering \normalfont\scshape} % Make all sections centered, the default font and small caps

\usepackage{lastpage}
\usepackage{fancyhdr} % Custom headers and footers
\pagestyle{fancyplain} % Makes all pages in the document conform to the custom headers and footers
\fancyhead{} % No page header - if you want one, create it in the same way as the footers below
\fancyfoot[L]{} % Empty left footer
\fancyfoot[C]{} % Empty center footer
\fancyfoot[C]{\thepage~of~\pageref{LastPage}} % Page numbering for right footer
\renewcommand{\headrulewidth}{0pt} % Remove header underlines
\renewcommand{\footrulewidth}{0pt} % Remove footer underlines
\setlength{\headheight}{13.6pt} % Customize the height of the header

\setlength\parindent{0pt} % Removes all indentation from paragraphs - comment this line for an assignment with lots of text

%----------------------------------------------------------------------------------------
%	TITLE SECTION
%----------------------------------------------------------------------------------------

\newcommand{\horrule}[1]{\rule{\linewidth}{#1}} % Create horizontal rule command with 1 argument of height

\usepackage{paralist} % Inline list environment



\title{
    \vspace{-1in} \usefont{OT1}{bch}{b}{n}
    \Large \bfseries \strut TDT4136 -- Introduction to Artificial Intelligence\\Assignment 1 -- A.I. Fundamentals and Intelligent Agents\strut \\
}

\begin{document}
\maketitle

\section*{Theoretical Questions}

\begin{enumerate}
\item \textbf{What is the Turing Test, and how is it conducted?}

The Turing Test, originally intruduced as ``The Imitation Game'' by Alan~M.~Turing in his paper ``Computing Machinery and Intelligence'' from 1950, describes a experiment intended to answer the question of whether machines can think. In the experiment, a machine (\textbf{A}) and a human (\textbf{B}) communicates purely textually with a human interrogator (\textbf{C}). Through conversation with both participants, the goal of the interrogator is to discern which participant is human. The objective of the human participant (\textbf{B}) is to help the interrogator conclude that it really is \textbf{B} that is human, while the machine participant is free to choose whichever strategy to convince the interrogator that it is the human.

No specific time limit was set for the experiment, but Turing estimated that a machine would by the year 2000 be able to play the game sufficiently well such that an average interrogator would not have more than a 70\% chance of correctly identifying the human participant after 5 minutes of playing. Turing did not describe a machine succeeding in the game as ``passing a test'', but described the experiment as a hypothetical and practical exercise to show that even if a machine might operate differently from a human, it is still able to think. 

\item \textbf{What is the relationship between thinking rationally and acting rationally? Is rational thinking an absolute condition for acting rationally?}
Whateverz.


\item \textbf{What is Tarski's ``theory of reference'' about?}
Alfred Tarski's ``theory of reference'' shows how to relate the objects in a logic to objects in the real world.

\item \textbf{Describe rationality. How is it defined?}

Rationality is defined as ``doing the right thing'' given the available information. Rationality describes the ability to base reasoning on logical inferences from available information and to act consistent with the available information to achieve goals established through reasoning. Within artificial intelligence a \textit{rational agent} will model the available information such that it can always perform the action available with the optimal expected outcome according to its goals. The goals of a rational agent will normally be modelled by utility functions which are used to calculate the expected outcome of different actions. Due to complexity it will often be impossible to always perform actions with optimal expected outcome, so in practise the term \textit{limited rationality} is used to describe acting appropriately when there is insufficient time to calculate the optimal answer.

\end{enumerate}

\end{document}

